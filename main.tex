\documentclass[a4paper]{report}
\usepackage{setspace}

\pagestyle{plain}
\usepackage{setspace}
\usepackage{hyperref}
\usepackage{amssymb}
\usepackage{graphicx}
\usepackage{color}
\usepackage{amsfonts}
\usepackage{latexsym}
\usepackage{amsmath}
\usepackage[toc,page]{appendix}
\setcounter{tocdepth}{1}
\usepackage{pdfpages}
\usepackage{todonotes}
\hypersetup{
    colorlinks,
    citecolor=black,
    filecolor=black,
    linkcolor=black,
    urlcolor=black
}

%\usepackage{etoolbox}
%\patchcmd{\abstract}{\titlepage}{}{}{}
%\patchcmd{\endabstract}{\endtitlepage}{}{}

\usepackage[authoryear]{natbib}
\usepackage{algorithm}
\usepackage{algpseudocode}

% \usepackage{caption}
\usepackage{subcaption}
\usepackage{float}
\usepackage{lipsum}

\newtheorem{theorem}{THEOREM}
\newtheorem{lemma}[theorem]{LEMMA}
\newtheorem{corollary}[theorem]{COROLLARY}
\newtheorem{proposition}[theorem]{PROPOSITION}
\newtheorem{remark}[theorem]{REMARK}
\newtheorem{definition}[theorem]{DEFINITION}
\newtheorem{example}{Example}

\newcommand{\nats}{\mbox{\( \mathbb N \)}}
\newcommand{\rat}{\mbox{\(\mathbb Q\)}}
\newcommand{\rats}{\mbox{\(\mathbb Q\)}}
\newcommand{\reals}{\mbox{\(\mathbb R\)}}
\newcommand{\ints}{\mbox{\(\mathbb Z\)}}

\usepackage{listings}
\usepackage{color}

\definecolor{codegreen}{rgb}{0,0.6,0}
\definecolor{codegray}{rgb}{0.5,0.5,0.5}
\definecolor{codepurple}{rgb}{0.58,0,0.82}
\definecolor{codeblue}{rgb}{0, 0, 204}
\definecolor{backcolour}{rgb}{248,255,235}

\lstdefinestyle{mystyle}{
    backgroundcolor=\color{backcolour},   
    commentstyle=\color{codegreen},
    keywordstyle=\color{codeblue},
    numberstyle=\tiny\color{codegray},
    stringstyle=\color{codepurple},
    basicstyle=\ttfamily\footnotesize,
    breakatwhitespace=false,         
    breaklines=true,                 
    captionpos=b,                    
    keepspaces=true,                 
    numbers=left,                    
    numbersep=5pt,                  
    showspaces=false,                
    showstringspaces=false,
    showtabs=false,                  
    tabsize=2
}

\lstset{style=mystyle}
\usepackage[a4paper, margin = 3cm, bottom = 2.5cm]{geometry}

\newtheorem{theorem}{THEOREM}
\newtheorem{lemma}[theorem]{LEMMA}
\newtheorem{corollary}[theorem]{COROLLARY}
\newtheorem{proposition}[theorem]{PROPOSITION}
\newtheorem{remark}[theorem]{REMARK}
\newtheorem{definition}[theorem]{DEFINITION}
\newtheorem{example}{Example}

\newcommand{\nats}{\mbox{\( \mathbb N \)}}
\newcommand{\rat}{\mbox{\(\mathbb Q\)}}
\newcommand{\rats}{\mbox{\(\mathbb Q\)}}
\newcommand{\reals}{\mbox{\(\mathbb R\)}}
\newcommand{\ints}{\mbox{\(\mathbb Z\)}}

%%%%%%%%%%%%%%%%%%%%%%%%%%

% ========================================
% Title Page
% ========================================
\title{{\vspace{-14em} \includegraphics[scale=0.4]{Logos/ucl_logo.png}}\\
{{\vspace{2em} \Huge Review of the Inverse $\mathbb{Z}$-Transform in Discrete Option Pricing}}\\
{\large Final Year Project Report}\\
}
\date{Submission date: \today}
\author{Roman Ryan Karim\thanks{
{\bf Disclaimer:}
This report is submitted as part requirement for the MEng degree in Mathematical Computation at UCL. It is substantially the result of my own work except where explicitly indicated in the text. The report may be freely copied and distributed provided the source is explicitly acknowledged.}
\\ \\ Dr Carolyn Phelan
\\ \\ \\ \\ Department of Computer Science
\\ University College London
\\ \\
}


% ========================================
% Report
% ========================================
\begin{document}
 
\onehalfspacing
\maketitle
\begin{abstract}

\end{abstract}

% ========================================
% Contents
% ========================================
\tableofcontents
\setcounter{page}{1}

% ========================================
% Introduction
% ========================================
\chapter{Introduction}
\section{Motivation}
We attribute the first recorded option contract to that of the Greek philosopher Thales of Miletus \citep{thompson1994aristotle}. Believing that the olive harvest would be plentiful, Thales secured the right to use the olive presses at a low price. This agreement is considered a call option as it gives the holder the right, but not the obligation, to execute the contract at a predetermined price.

The formal study of pricing options began much later with \citet{bachelier1900theorie}'s work in modelling stock prices as a Brownian motion and his derivation of an option pricing formula. However, we deem the seminal work of \citet{black1973pricing} in laying out the foundation for modern option pricing theory. The introduction of the Black-Scholes model coincided with the establishment of the Chicago Board Options Exchange (CBOE) in 1973, which standardized option contracts and provided a platform for trading. The rapid growth (Fig \ref{fig:volume_of_options}) in the options market sparked a demand and increased interest in option pricing research.

\begin{figure}[h]
	\centering
	\includegraphics[width=0.7\linewidth]{images/options_volume.png}
    \caption{Visualization of the volume of options (in billions) traded over the past 17 years on exchanges CBOE, BATS, C2 and EDGX.}
    \label{fig:volume_of_options}
\end{figure}

However, the assumptions of the Black-Scholes model does not lend itself to an accurate representation of real-world asset price data. The price of the underlying asset is monitored at discrete points in time. One could say that in practice most, if not all, path-dependent options in markets are discrete path-dependent options \citep{kou2007discrete}. The pricing of discrete path-dependent options has been a longstanding area of research in the financial mathematics literature. The importance and value of these options results in a vast array of research including that around the Black-Scholes model assumption \citep{lu2017improved, guardasoni2020mellin}, general L\'evy processes \citep{fang2009novel, fusai2016spitzer, phelan2018fourier, chen2021sinc, levendorskii2022sinh}, general Markov processes \citep{cui2021pricing, zhang2023general} and stochastic volatility \citep{soleymani2019pricing, kirkby2020efficient}.

A common problem faced by those in the literature is the linear dependency on the monitoring dates of the underlying asset price. \citet{fusai2016spitzer} demonstrated that the iteration on the monitoring dates can be avoided by working in the $z$-domain, extending on work by that of \citet{carr1999option}. Applying the \citet{spitzer1957wiener} identities and solving the resulting equations requires an inverse operation to obtain the option price. Reverting from the $z$-domain to the time domain requires a numerical approximation of the inverse $z$-transform. Despite its relation to the Fourier transform, the literature lacks in comparison; the performance of existing approximations, in terms of accuracy and efficiency, necessitates a specific setup and choice of parameters.

\section{Aims and Objectives}
Given the importance of the numerical inverse $z$-transform (NIZT), this project focuses on \textbf{(i)} studying the effect of the parameters on the varying methods, and \textbf{(ii)} testing on a group of well-defined transform pairs. Thus, our aim is to try and discover which methods provide the best fit in terms of efficiency and accuracy and what is the best setup to achieve this. The measurable objectives (O1-O3) associated with this aim are as follows:

\begin{itemize}
	\item (O1) Analyze the existing NIZT methods in literature and the effects on parameter changes
	\item (O2) Find the optimal parameter setup for each method
	\item (O3) Test the implementations against well-defined transform pairs
\end{itemize}

\section{Overview}
The upcoming chapters of this report aims to provide a detailed description of the different stages taken through this project. In Chapter 2, we provide a self-contained technical background on the key concepts and methods used in this project. Given the breadth of the topic, references are included for those looking to delve deeper into the subject. In Chapter 3, we outline the experiments conducted in Chapter 4 to evaluate the performance of the numerical approximation methods. We detail the implementation of the methods and the experimental setup, including parameter selection and the benchmarking process We then provide the results on a list of transform pairs and discuss the implications of the findings. Finally, Chapter 5 summarises the steps taken and the findings of this project with suggested work for further continuation. 

% ========================================
% Background
% ========================================
\chapter{Background}

In Chapter 2, we establish a foundational understanding of the topic in hand. This section is designed to be self-contained, providing essential background for all readers, while references are included for those seeking a deeper exploration.

We conduct an analysis into the mathematical concepts used in option pricing, with a focus on Fourier-based methods. While a large body of theoretical work exists in this area, the practical implementation of these methods is often computationally expensive, especially when evaluating high-dimensional integrals involved in the pricing formulas. We study literature on numerical approximation methods and the application in the pricing of exotic options, specifically lookback and barrier options. Subsequently, we explore the use-case of machine learning techniques to optimize parameters during our experimentation in Chapter 3.

% ========================================
% The Z-Transform
% ========================================
\section{The \texorpdfstring{$\mathcal{Z}$}{Lg}-Transform}\label{z_transform}

The $z$-transform is a transformation of a real or complex time function $x(n)$, often used for analyzing discrete-time signals and systems. It is a generalization of the discrete-time Fourier transform (DTFT) that extends the analysis to the complex plane. The Z-transform is formally defined as:

\begin{equation}\label{bilateral_z-transform}
X(z) = \mathcal{Z}_{n \rightarrow z}[x(n)] = \sum^{\infty}_{n = -\infty} x(n)z^{-n}
\end{equation}
``By definition, a complex number $z$ is an ordered pair $(x, y)$ of real numbers $x$ and $y$, written $z = (x, y)$'' \citep{kreyszig2010advanced}. In practice, complex numbers are written in the form $z = x + iy$, where $x$ and $y$ are real numbers and $i$ is the imaginary unit. We may find it easier to represent complex numbers in their polar form, $z = re^{i\theta}$, where $r$ represents the magnitude of $z$ and $\theta$ represents the angle of $z$ with respect to the positive real axis.

In the analysis of causal systems - systems for which a time origin is defined and is illogical to consider signal values for negative time - the unilateral $z$-transform is used. Unlike the bilateral $z$-transform in Eq. (\ref{bilateral_z-transform}), we sum from zero to positive infinity yielding

\begin{equation}\label{unilateral_z-transform}
X(z) = \mathcal{Z}_{n \rightarrow z}[x(n)] = \sum^{\infty}_{n = 0} x(n)z^{-n}
\end{equation}

The region within the complex $z$-plane where the $z$-transform converges is known as the Region of Convergence (ROC). The ROC is defined for the set of values of $z$ for which the $z$-transform is absolutely summable

\begin{equation}\label{roc}
\textbf{ROC} = \Biggl\{ z : \sum^{\infty}_{n = 0} |x(n)z^{-n}| < \infty \Biggr\}
\end{equation}

For causal sequences, the ROC is typically the exterior of the outermost pole in the $Z$-plane, denoted as $|z| > a$. If we say that $z_1$ converges, then $z_1$ exists within the ROC. Thus, all $z$ such that $|z| \geq |z_1|$ also converge. This region excludes the poles themselves, as the transform does not converge at those points. For the system to be $stable$, the ROC must include the unit circle, $|z| = 1$, implying that all poles must lie within the unit circle \citep{LovelessGuido2021}.

\begin{example}\label{example:roc_poles}
    Consider the function given by \todo{keep here or in Section 3.1?}
    
    \begin{equation}
        H(z) = \frac{(z - i)(z + i)}{\left(z - \left(-\frac{1}{4} - \frac{1}{2}i\right)\right)\left(z - \left(\frac{1}{2} + \frac{1}{2}i\right)\right)}
    \end{equation}
    
    The ROC of $H(z)$ must exclude the poles at $z = -\frac{1}{4} - \frac{1}{2}i$ and $z = \frac{1}{2} + \frac{1}{2}i$, thus the ROC is $|z| > \frac{1}{2}$. The system is stable as the ROC includes the unit circle.

\end{example}

% ========================================
% Relation to Fourier Transform
% ========================================
\subsection{Relation to the Fourier Transform}\label{rs_fourier_transform}
It is useful to note the relationship between the $z$-transform and the Fourier transform. Taking the Fourier transform of a sampled function $x(t)$ results in:

\begin{flalign}
&& \mathcal{F}\bigg[x(t) \sum^{\infty}_{n = -\infty} \delta(t - n \Delta t)\bigg] &= \int^{\infty}_{-\infty} x(t) \sum_{n = - \infty}^{\infty} \delta (t - n\Delta t)e^{-i \omega t} dt && \\
&& &= \sum_{n = - \infty}^{\infty} \int^{\infty}_{-\infty} x(t) \delta (t - n\Delta t)e^{-i \omega t} dt && \\
&& &= \sum^{\infty}_{n=-\infty} x(n \Delta t)e^{-i \omega nt} &&
\end{flalign}

where we make use of the sifting property of the delta function. If we normalize the sampling interval to 1, we get

\begin{equation}\label{dtft}
\sum^{\infty}_{n = - \infty} x(n)e^{-i n \omega}
\end{equation}

This is the discrete-time Fourier transform (DTFT) of the sequence $x(n)$. The sequence $x(n)$ is sampled at discrete-time intervals $t_n = n \triangle t$, where the sampling interval $\triangle t$ is the time between consecutive samples and the time index $n$ numbers the samples. The DTFT is a periodic function of $\omega$ with period $2\pi$, and its existence relies on the absolute summability of the sequence $x(n)$:

\begin{equation}
\sum^{\infty}_{n = -\infty} |x(n)| < \infty
\end{equation}

The Z-transform generalizes Eq. (\ref{dtft}) to the complex plane, not just the unit circle where $r = 1$ \citep{Oppenheim1989DTSP}.

% ========================================
% Relation to PDFs
% ========================================
\subsection{Relation to Probability Distribution Functions}\label{pdfs}
Random events and signals refer to situations where the outcome is not deterministic, but can be described by probability. Understanding these concepts involves using Probability Distribution Functions; the Probability Mass Function (PMF) for discrete random variables and the Probability Density Function (PDF) for continuous random variables. Given the nature of this project, we'll be focusing our attention on the PMF.

The PMF is defined for a discrete random variable $X$ taking on values $x_i$ with probabilities $p_i$, as $P(X=x_i) = p_i$. The PMF satisfies the following properties:

\begin{equation}
    \sum_{i=0}^n p_i = 1 \hspace{2em}\text{and}\hspace{2em} 0 \leq p_i \leq 1 \hspace{0.5em}\forall i
\end{equation}

We may find it useful to provide a concise representation of the entire distribution such that we expand upon the PMF, $p(x)$, to obtain the Probability Generating Function (PGF), $G_X(q)$, defined as

\begin{equation}
	G_X(q) = E[q^X] = \sum^{\infty}_{x = 0} p(x)q^x,
\end{equation}
where $E[\cdot]$ denotes the expectation operator, and $q$ is a complex number. We deliberately use $q$ to distinguish the PGF from the $z$-transform (Eqn. \ref{z_transform}).

% Example using a fair dice
\begin{example}
    Consider a fair six-sided dice. The PMF for the dice roll is given by
    
    \begin{equation}
        p(x) = \begin{cases}
            \frac{1}{6} & \text{if } x = 1, 2, 3, 4, 5, 6 \\
            0 & \text{otherwise}
        \end{cases}
    \end{equation}

    where $p(x)$ is the probability of rolling a number $x$. The PGF for the dice roll is then

    \begin{equation}
        G_X(q) = \sum^{\infty}_{x = 0} p(x)q^x = \frac{1}{6} \sum^6_{x = 1} q^x = \frac{q}{6}\cdot \frac{1-q^6}{1-q},
    \end{equation}

    where we use the formula for the sum of a geometric series. The PGF encapsulates the entire distribution of the dice roll into a single function.
\end{example}

The concept of summarizing information is not unique to probability theory. In the analysis of signals, we aim to encapsulate the behaviour of a sequence into a single function. This is akin to the PGF, where the $z$-transform is used to analyze discrete-time signals and systems. Drawing on the principles outlined by \citet{ross2014introduction}, we can bridge the gap between probability theory and signal processing, leveraging the $z$-transform to analyze the behaviour of signals in the complex plane.

% ========================================
% The Inverse Z-Transform
% ========================================
\section{The Inverse \texorpdfstring{$\mathcal{Z}$}{Lg}-Transform}

The inverse $Z$-transform aims to find the $n$-th value of the sequence $x(n)$ given the $Z$-transform $X(z)$. This is commonly defined as a Cauchy integral around a contour $C$ in the complex plane. The contour $C$ is a counter-clockwise closed path that encloses the region of convergence (ROC). The inverse $Z$-transform is formally given by

\begin{equation}\label{inverse_z-transform}
	x(n) = \mathcal{Z}^{-1}_{z \rightarrow n}[X(z)] = \frac{1}{2\pi i} \oint_C X(z)z^{n-1}dz
\end{equation}

In practical settings, numerical approximations are often used due to the computational challenges of evaluating the Cauchy integral. Whilst there are many ways to go about this \citep{merrikh2014linearsys,rajkovic2004method,horvath2020numerical}, most methods are done on a circular contour. Aligning with the focus of our project, we shift our attention to contour integrals.

% MAYBE: include other methods if we have time

% ========================================
% Abate and Whitt 1992
% ========================================
\subsection{Abate and Whitt 1992}\label{abate_whitt_section}
The numerical approximation formula offered by \citet{AbateWhitt1992a, AbateWhitt1992b} is based on a Fourier series catering to the inversion of probability generating functions as elucidated in Section \ref{pdfs}. The format is conducive to queuing theory and other probabilistic models where the $Z$-transform is defined as $q = 1 / z$. The authors approximate the inversion using a trapezoidal rule for numerical integration over a complex contour given by

\begin{equation}\label{aw_inversion_original}
	x(n) \approx \frac{1}{2nr^n} \biggr( X(r) + 2\sum^{n-1}_{k = 1} (-1)^k \mathrm{Re}\bigg( X(re^{\frac{ik\pi}{n}})\bigg) + (-1)^nX(-r) \biggl)
\end{equation}

The parameter $r$ is used to control the error; setting $r = 10^{-\lambda / 2n}$ yields an accuracy bound of $10^{-\lambda}$. The authors leverage the inherent symmetry within the complex plane to enhance computational efficiency by exploiting the complex conjugate symmetry of $X(z)$; each term $X(re^{\frac{ik\pi}{n}})$ in the upper half has a mirror image in the lower half. The computational load is thus halved by \textit{folding} the problem in this manner.

Given the nature of this project, we may find it easier to use the following definition, where we set $z = 1 / q$, to approximate Eq. (\ref{inverse_z-transform}).

\begin{equation}\label{aw_inversion}
	x(n) \approx \frac{1}{2nr^n} \biggr( X(\frac{1}{r}) + 2\sum^{n-1}_{k = 1} (-1)^k \text{Re}\bigg( X\big(\frac{1}{re^{\frac{ik\pi}{n}}}\big)\bigg) + (-1)^nX(-\frac{1}{r}) \biggl)
\end{equation}

The Nyquist-Shannon sampling theorem states that a signal must be sampled at a rate of at least twice the highest frequency present in the signal to avoid aliasing \citep{shannon1949communication,nyquist1928certain}. The number of points, $n$, used in Eqn. (\ref{aw_inversion}) must be sufficient to capture the information properly. If $n$ is too small, the approximation may lead to inaccuracies - akin to aliasing in signal processing.

% ========================================
% Cavers 1978
% ========================================
\subsection{Cavers 1978}\label{cavers_section}
Extending upon our analysis in Section \ref{rs_fourier_transform}, \citet{Cavers1978FFT} proposes to sample the $z$-transform of a function on a circular contour at equally spaced points and then apply the inverse FFT to these sampled points to approximate the original time-domain signal. We can formulate this as:

\begin{equation}\label{cavers}
	f(n) = r^n \text{IFFT}[f(re^{2\pi i / N})],
\end{equation}

where $r$ is the radius of the circular contour, $n$ is the time index, and $N$ is the number of points used in the DFT. The factor $r^n$ scales the result appropriately based on the radius of the contour.

\subsection{Series acceleration techniques}
\todo{if we have time, otherwise remove this section}

\section{Optimization Techniques}
In the context of computational mathematics, optimizations techniques are used to identify the optimal or a sufficiently effective solution to a problem within a given set of constraints. The goal is to minimize or maximize a specific objective function by systematically choosing the values of the variables. The objective function is often referred to as the \textit{cost function} or \textit{loss function} and the variables are referred to as \textit{parameters}. The optimization problem can be formulated as

\begin{equation}\label{optimization_problem}
	\text{minimize } f(x) \text{ subject to } x \in \Omega,
\end{equation}
where $f(x)$ is the objective function and $\Omega$ is the feasible region defined by the constraints of the problem.

Gradient descent is one of the most popular algorithms for parameter optimization with success in Deep Learning and Neural Networks employing variants of the algorithm \citep{lu2017improved, zhang2019gradient, zeebaree2019trainable}. The adaptability to diverse problem domains \citep{YingjieYugiHaibin2023SGD} parallels our use case, where gradient descent is applied outside traditional deep learning to optimize parameters of a mathematical function \citep{GradientBasedOpt2022}.

% ========================================
% Gradient Descent
% ========================================
\subsection{Gradient Descent}
Gradient descent iteratively converges to a local minimum of a function by moving in the direction of the steepest descent, as defined by the negative gradient. This method is expressed mathematically as

\begin{equation}\label{GD}
    x_{k+1} = x_k - \alpha_k \nabla f(x_k),
\end{equation}
where $x_k$ is the parameter vector at iteration $k$, $\alpha_k$ is the learning rate, and $\nabla f(k)$ represents the gradient of the function at $x_k$. The selection of $\alpha_k$ determines the size of the step taken towards the minimum; too large can overshoot the minimum, too small can result in a long convergence time. The process repeats until a predetermined termination criterion is met, typically when the change in the value of $f(k)$ falls below a threshold. This iterative process is showcased in the pseudocode below:

\begin{algorithm}
\caption{Gradient Descent}
\begin{algorithmic}[1]
\State Initialize \( x_0 \), set \( k = 0 \)
\While{termination conditions not met}
    \State Compute gradient \( \nabla f(x_k) \)
    \State Choose a suitable step size \( \alpha_k \)
    \State Update \( x_{k+1} = x_k - \alpha_k \nabla f(x_k) \)
    \State \( k = k + 1 \)
\EndWhile
\end{algorithmic}
\end{algorithm}

% ========================================
% Stochastic Gradient Descent
% ========================================
\subsubsection{Stochastic Gradient Descent}

However, classic Gradient Descent faces limitations, including susceptibility to local minima and potential for overshooting or long convergence times. Stochastic Gradient Descent (SGD) addresses these issues by introducing variability in the optimization process. It modifies Eq. (\ref{GD}) to use a randomly selected subset of data to compute the gradient, to allow for dynamic adjustment of the learning rate and leveraging noise to escape local minima. We define the update rule to 

\begin{equation}\label{SGD}
x_{k+1} = x_k - \alpha_k \nabla f_{i_k}(x_k)	
\end{equation}
where $\nabla f_{i_k}(x_k)$ is the gradient of the cost function with respect to a random subset $i_k$. We thus avoid the pitfalls associated with a static learning rate and promote a quicker convergence time.

\section{Option Pricing}\label{section:option_pricing}
The concept involves determining the value of options, which are financial contracts that give the holder the right, \textit{but not the obligation}, to buy or sell an asset at a set price within a specified time-frame. The value of an option is derived from the underlying asset, which can be a stock, bond, or commodity. 

A pivotal point in option pricing was the introduction of \citet{black1973pricing}'s model in estimating the price of European-style options, which can only be exercised at the expiration date. The Black-Scholes model is based on the assumption that the price of the underlying asset follows a geometric Brownian motion in idealistic conditions. However, many options traded in real markets are American-style, allowing the holder to exercise the option at any time before the expiration date. This complicates the pricing process as it involves solving an \textit{optimum stopping problem}. \citet{merton1973theory} extends the Black-Scholes model to American options by expressing the price as the solution to a free boundary problem. While Merton's work provided a theoretical foundation, solving the free boundary problem analytically is challenging. Instead, numerical methods such as binomial trees \citep{cox1979option} and simulation-based methods \citep{longstaff2001simulation} have been developed to price American options.

\subsection{Discrete Monitoring}\label{section:discrete_monitoring}
In most cases, the payoff of an option depends on the price of the underlying asset at discrete points in time rather than continuously. This is known as discrete monitoring. Two widely traded types of discretely monitored options are lookback and barrier options \citep{dadachanji2015fx}. These options are classified as \textit{exotic options} where the payoff is based on the path of the underlying asset price rather than just the price at expiration.

\subsubsection{Lookback Options}
\begin{figure}[H]
    \begin{subfigure}{.5\linewidth}
      \includegraphics[width=\linewidth]{images/call_option.png}
      \caption{Lookback Call Option}
      \label{fig:call_option}
    \end{subfigure}\hfill
    \begin{subfigure}{.5\linewidth}
      \includegraphics[width=\linewidth]{images/put_option.png}
      \caption{Lookback Put Option}
      \label{fig:put_option}
    \end{subfigure}
    \caption{Illustration of Lookback Options: Displaying the mechanics of (a) Call Option and (b) Put Option. The red line represents the barrier, while the red dot marks the strike price.}
\end{figure}

\noindent The payoff depends on the maximum and minimum asset price over the life of the option. A \textit{lookback call option} (\autoref{fig:call_option}) gives the holder the right to buy the asset at the lowest price during the option period, while a \textit{lookback put option} (\autoref{fig:put_option}) allows the holder to sell the asset at the highest price during the option period.

\subsubsection{Barrier Options}
\begin{figure}[H]
    \begin{subfigure}{.5\linewidth}
      \includegraphics[width=\linewidth]{images/up_in_option.png}
      \caption{Up-and-In Barrier Option}
    \end{subfigure}\hfill
    \begin{subfigure}{.5\linewidth}
      \includegraphics[width=\linewidth]{images/up_out_option.png}
      \caption{Up-and-Out Barrier Option}
    \end{subfigure}
    
    \medskip
    \begin{subfigure}{.5\linewidth}
      \includegraphics[width=\linewidth]{images/down_in_option.png}
      \caption{Down-and-In Barrier Option}
      \label{fig:down_in_option}
    \end{subfigure}\hfill
    \begin{subfigure}{.5\linewidth}
      \includegraphics[width=\linewidth]{images/down_out_option.png}
      \caption{Down-and-Out Barrier Option}
      \label{fig:down_out_option}
    \end{subfigure}
    
    \caption{Illustration of Barrier Options: Displaying the active (black) and inactive (grey) phases of each option type. (a) Up-and-In Option, (b) Up-and-Out Option (c) Down-and-In Option, and (d) Down-and-Out Option. The red line represents the barrier.}
\end{figure}

The payoff depends on whether the price of the underlying asset reaches a certain level (the barrier) during the life of the option. A \textit{knock-in} barrier option only come into existence if the barrier has been touched, while a \textit{knock-out} barrier option ceases to exist instead. For example, a \textit{down-and-out} barrier option (\autoref{fig:down_out_option}) is a type of knock-out option that becomes worthless if the price of the underlying asset falls below the barrier level. On the other hand, a \textit{down-and-in} barrier option (\autoref{fig:down_in_option}) is a type of knock-in option that only becomes active if the asset price falls below the barrier level.

\subsection{Modelling Asset Prices}

\subsection{Fourier-Based Pricing}
However, the pricing of these options can be computationally expensive due to the high-dimensional integrals involved in the pricing formulas. Whilst \citet{heston1993closed} explored the idea of modelling asset prices in the Fourier domain, \citet{carr1999option} were the first to price European options expressing both the characteristic function and the payoff in the Fourier domain. It involved transforming the pricing problem from the time domain to the frequency domain, where the pricing formula simplifies to a one-dimensional integral. The Fourier-based methods have been widely used for the pricing of options discussed in Section \ref{section:discrete_monitoring} \citep{eberlein2010analysis}.

\citet{fang2009novel} introduced a pricing technique based on the Fourier-cosine expansion approximating the characteristic function of the underlying asset price, which was then expanded to a broader class of exotic options \citep{fang2009pricing, fang2011fourier} and general l\'evy processes \citep{lord2008fast}. Whilst the method shows high levels of accuracy with exponential error convergence on the number of terms, this is only the case when the governing probability density function is sufficiently smooth. When modelling the asset price by the variance gamma process \citep{madan1998variance}, we achieve only algebraic error convergence due to the discontinuities in the process. \citet{ruijter2015application} showed that this can be remedied using spectral filters however the issue remains in that the computational time is linearly dependent on the number of monitoring dates for discretely monitored options. This seems to also be the case for \citet{feng2008pricing}'s employment of the Hilbert transform using backward induction to price barrier options. 

In an effort to eliminate the computational inefficiency tied to the number of monitoring dates, \citet{fusai2006exact} looked into shifting the pricing problems from the time domain into the $z$-domain, where convolution operations become straightforward multiplications. This helped set the stage for \citet{fusai2016spitzer}'s method incorporating the Hilbert and $z$-transform to calculate the \textbf{Wiener-Hopf factors}, which decompose the characteristic function of the price process into two parts; positive and negative jumps. The authors demonstrated that the computational cost is independent of the number of monitoring dates, and the error decays exponentially with the number of grid points. Whilst extensions to the method have been made such as those by \citet{phelan2018fourier}'s new scheme using spectral filters to improve convergence \citep{phelan2019hilbert}, we shift our focus to the use-case of the inverse $z$-transform in the pricing of options.

\subsection{NIZT in Option Pricing}
\begin{itemize}
    \item calculating characteristic function's $z$-transform - PD of asset's price
    \item IZT to get the PD of the asset's price at different points
    \item use of the PD to calculate the option price
\end{itemize}
% ========================================
% Experiment
% ========================================
\chapter{Experiment}
Having established the theoretical foundation, we now turn our attention to the practical implementation of methods proposed by \citet{AbateWhitt1992a, AbateWhitt1992b} and \citet{Cavers1978FFT}. Whilst the former is based on a circular contour, we also explore the idea laid out by \citet{levendorskii2022sinh} in sampling the $z$-transform on a sinh-deformation.

This chapter outlines the experimental setup, implementation and evaluation of the methods. We measure our results in terms of accuracy and computational efficiency against well-known transformation pairs in an attempt to provide a thorough comparison of the methods to determine the most effective approach.
\todo{If we get series acceleration, include here asw}

\section{Circular Contour}
``The inverse at point $T$ can be obtained from the contour integral where $C$ is a counter-clockwise closed path encircling the origin and entirely in the region of convergence'' \citep{horvath2020numerical}. A circular contour is commonly used to solve Equation \ref{inverse_z-transform}, defined as $z = re^{i\theta}$, such that $r$ encloses the poles of the $z$-transform. 

\begin{figure}[H]
    \begin{subfigure}{.25\linewidth}
      \includegraphics[width=\linewidth]{images/invalid_contour.png}
      \caption{r = 0.5}
    \end{subfigure}\hfill
    \begin{subfigure}{.25\linewidth}
      \includegraphics[width=\linewidth]{images/valid_border_contour.png}
      \caption{r = 0.72 (2 d.p.)}
    \end{subfigure}\hfill
    \begin{subfigure}{.25\linewidth}
      \includegraphics[width=\linewidth]{images/unstable_contour.png}
      \caption{r = 1.5}
    \end{subfigure}\hfill
    
    \caption{ Displaying the effect of the radius $r$ on a circular contour ($n = 100$). The red dots represent the poles of $H(z)$ in Example \ref{example:roc_poles} with (a) including the poles within the ROC, (b) excluding the poles but inclusive of the unit circle, and (c) excluding the poles but not stable.}
\end{figure}

\subsection{Abate and Whitt 1992}

\begin{algorithm}[H]
    \caption{Implementation of Equation \ref{aw_inversion}}
    \begin{algorithmic}[1]
    \Procedure{AbateWhitt}{$\tilde{f}, n$}
        \State $\lambda \gets x_\in \mathbb{Z}^+$
        \State $\rho \gets 10^{-\lambda / (2 \cdot n)}$
        \State $\text{summation} \gets 0$
        \For{$k \gets 1$ \textbf{to} $n-1$}
            \State $z \gets \frac{1}{\rho \cdot \exp(\text{i} \cdot k \cdot \pi / n)}$
            \State $\text{sum} \gets \text{sum} + (-1)^k \cdot \text{Re}(\tilde{f}(z))$
        \EndFor
        \State $\text{sum} \gets \tilde{f}(\frac{1}{\rho}) + 2 \cdot \text{sum} + (-1)^n \cdot \tilde{f}(\frac{1}{-\rho})$
        \State \Return $\frac{\text{sum}}{2 \cdot n \cdot \rho^n}$
    \EndProcedure
    \end{algorithmic}
\end{algorithm}

\subsection{Cavers 1978}

\section{Sinh Deformation}
A circular contour integral is commonly used for approximating the inverse $z$-transform as we've seen in Section \ref{abate_whitt_section} and \ref{cavers_section}. However, this does not mean it may be the best option. \citet{levendorskii2022sinh} propose a new method for a numerical evaluation of Equation \ref{inverse_z-transform} by deforming the commonly used circular contour $\{z = re^{i\theta} | -\pi < \theta < +\pi\}$ through the conformal mapping:
\begin{equation}\label{equation:conformal_mapping}
    \xi(y) = \sigma + ib\sinh(i\theta + y),
\end{equation}
where parameters are chosen such that $\sigma_\in \mathbb{R}$, $b > 0$, and $\theta$ is restricted to the interval $(-\pi/2, \pi / 2)$. The unique selection of parameters allows the contour to be tailored specifically to the characteristics of $X(z)$. The authors state that the conformal mapping alleviates typical errors which are discussed in the relevant texts \citep{boyarchenko2014efficient, boyarchenko2019sinh, schmelzer2007computing}. Applying a change of variables to Equation \ref{inverse_z-transform} yields

\begin{equation}
    x(n) = \int_\mathbb{R} \frac{b}{2\pi} \xi(y)^{-n-1} \cosh(i\theta + y) X(\xi(y)) dy.
\end{equation}

By denoting the integrand as $f_n(y)$ and approximating the application of the infinite trapezoidal, we have 
\begin{equation}
    x(n) \approx 2 \epsilon\ \text{Re}\left( \sum_{j = 0}^{M_0} f_n(j \epsilon)(1 - \delta_{j0}/2) \right)
\end{equation}

\todo[inline]{need to look more into it}

\begin{itemize}
    \item mention sufficient conditions under which this works (next paper)
    \item talk about why he does this
\end{itemize}

\subsection{Deforming the Contour}
In an attempt to recreate the unit circle from the conformal mapping (\ref{equation:conformal_mapping}), we first try to understand the effects of the parameters on the contour. We plot the contour for different values of $\sigma, b$ and $y$ to observe the deformation.

\begin{figure}[ht]
    \begin{subfigure}{.3\linewidth}
      \includegraphics[width=\linewidth]{images/deformations/base.png}
      \caption{$\sigma = 0, b = 0.5, y = 1.0$}
      \label{fig:base_deform}
    \end{subfigure}\hfill
    \begin{subfigure}{.3\linewidth}
      \includegraphics[width=\linewidth]{images/deformations/positive_sigma.png}
      \caption{$\sigma + 0.2$}
      \label{fig:positive_sigma}
    \end{subfigure}\hfill
    \begin{subfigure}{.3\linewidth}
      \includegraphics[width=\linewidth]{images/deformations/negative_sigma.png}
      \caption{$\sigma - 0.2$}
        \label{fig:negative_sigma}
    \end{subfigure}

    \medskip

    \begin{subfigure}{.3\linewidth}
        \includegraphics[width=\linewidth]{images/deformations/higher_b.png}
        \caption{$b + 0.2$}
        \label{fig:higher_b}
      \end{subfigure}\hfill
      \begin{subfigure}{.3\linewidth}
        \includegraphics[width=\linewidth]{images/deformations/lower_b.png}
        \caption{$b - 0.2$}
        \label{fig:lower_b}
      \end{subfigure}\hfill
      \begin{subfigure}{.3\linewidth}
        \includegraphics[width=\linewidth]{images/deformations/higher_y.png}
        \caption{$y + 0.2$}
        \label{fig:higher_y}
      \end{subfigure}

      \medskip

    \begin{subfigure}{.3\linewidth}
        \includegraphics[width=\linewidth]{images/deformations/lower_y.png}
        \caption{$y - 0.2$}
        \label{fig:lower_y}
      \end{subfigure}\hfill
      \begin{subfigure}{.3\linewidth}
        \includegraphics[width=\linewidth]{images/deformations/mixing_b_y.png}
        \caption{$b - 0.1, y + 0.2$}
        \label{fig:mixing_b_y}
      \end{subfigure}\hfill
      \begin{subfigure}{.3\linewidth}
        \includegraphics[width=\linewidth]{images/deformations/mixing_b_y_v2.png}
        \caption{$b + 0.1, y - 0.2$}
        \label{fig:mixing_b_y_v2}
      \end{subfigure}
    
    \caption{Displaying the effect of the parameters on the contour for the conformal mapping (\ref{equation:conformal_mapping}) with $\{-\frac{\pi}{2} \leq \theta \leq \frac{\pi}{2} \}$ and $n = 100$ points. Plots (b, c, d, e, f, g, h, i) highlights the changes in parameter the effects on the contour in comparison to the base contour (a). The black line represents the created contour, with the grey contour representing the base contour with parameters $\sigma = 0, b = 0.5, y = 1.0$.}
\end{figure}

In relation to the base contour (\autoref{fig:base_deform}), we observe that increasing $\sigma$ shifts the contour to the right (\autoref{fig:positive_sigma}), while decreasing $\sigma$ shifts the contour to the left (\autoref{fig:negative_sigma}). The parameter $b$ acts as a scale factor on the sinh deformation as shown in \autoref{fig:higher_b} and \ref{fig:lower_b}. Similarly, $y$ acts as a scale factor but with a more pronounced vertical effect as seen in \autoref{fig:higher_y} and \ref{fig:lower_y}. Mixing the parameters $b$ and $y$ results in a more complex deformation where the counter-opposing effects changes the contour in a non-linear manner (\autoref{fig:mixing_b_y} and \ref{fig:mixing_b_y_v2}).

\subsection{Parameter Grid Search}
\begin{itemize}
    \item use of grid search to find optimal parameters
    \item different search criteria's (\textit{e.g. area, x-axis perimeter, etc})
    \item might also be useful to try find parameters that give the results?
    \item use of SGD and loss function
\end{itemize}

\section{Numerical Benchmarking}

\begin{table}[h]
    \centering
    \begin{tabular}{c|cc}
    \hline
    \textbf{Function} & $x(z)$ & $X(z)$ \\
    \hline
    Heaviside Step & 1 & $z / \textbf{(}z-1\textbf{)}$ \\
    Polynomial & \(t\) & $z / \textbf{(}z-1\textbf{)}^2 $ \\
    Decaying Exp & $e^{-at}$ & $1 / \textbf{(}1 - \exp(a\triangle t) z^{-1}\textbf{)}$ \\
    Sinusoidal & $\sin(\omega t)$ & $\textbf{(}z^{-1}\sin(w\triangle t)\textbf{)} / \textbf{(}1-2z^{-1}\cos(w\triangle t) + z^{-2}\textbf{)}$
    \end{tabular}
    \caption{List of Transform Pairs}
    \label{tab:transform_pairs}
\end{table}

% ========================================
% Results
% ========================================
\chapter{Results}

\section{Sinh Deformation}

\subsection{Deforming the Contour}

\subsection{Parameter Search}

% ========================================
% Conclusion
% ========================================
\chapter{Conclusion}
\section{Summary}

\section{Future Work}

\subsection{Parameter Selection}
During the course of the project, there was no set method for selecting the parameters for Equation \ref{equation:conformal_mapping}, hence, our effort of employing numerous techniques in an attempt to find the optimal parameters. However, as of recently at the time of writing, \citet{boyarchenko2024efficient} have released a paper detailing parameter selection for the conformal mapping (\ref{equation:conformal_mapping}).

\section{Acknowledgements}


% ========================================
% References 
% ========================================
\addcontentsline{toc}{chapter}{References}
\bibliography{references}
\bibliographystyle{apalike}

% ========================================
% Appendix
% ========================================
\begin{appendices}

% \chapter{Initial Project Plan}
% \includepdf[pages=-]{initial_project_plan.pdf}
    
\end{appendices}

\end{document}