\documentclass[a4paper]{report}
\usepackage{setspace}

\pagestyle{plain}
\usepackage{setspace}
\usepackage{hyperref}
\usepackage{amssymb}
\usepackage{graphicx}
\usepackage{color}
\usepackage{amsfonts}
\usepackage{latexsym}
\usepackage{amsmath}
\usepackage[toc,page]{appendix}
\setcounter{tocdepth}{1}
\usepackage{pdfpages}
\usepackage{todonotes}
\hypersetup{
    colorlinks,
    citecolor=black,
    filecolor=black,
    linkcolor=black,
    urlcolor=black
}

%\usepackage{etoolbox}
%\patchcmd{\abstract}{\titlepage}{}{}{}
%\patchcmd{\endabstract}{\endtitlepage}{}{}

\usepackage[authoryear]{natbib}
\usepackage{algorithm}
\usepackage{algpseudocode}

% \usepackage{caption}
\usepackage{subcaption}
\usepackage{float}
\usepackage{lipsum}

\newtheorem{theorem}{THEOREM}
\newtheorem{lemma}[theorem]{LEMMA}
\newtheorem{corollary}[theorem]{COROLLARY}
\newtheorem{proposition}[theorem]{PROPOSITION}
\newtheorem{remark}[theorem]{REMARK}
\newtheorem{definition}[theorem]{DEFINITION}
\newtheorem{example}{Example}

\newcommand{\nats}{\mbox{\( \mathbb N \)}}
\newcommand{\rat}{\mbox{\(\mathbb Q\)}}
\newcommand{\rats}{\mbox{\(\mathbb Q\)}}
\newcommand{\reals}{\mbox{\(\mathbb R\)}}
\newcommand{\ints}{\mbox{\(\mathbb Z\)}}

\usepackage{listings}
\usepackage{color}

\definecolor{codegreen}{rgb}{0,0.6,0}
\definecolor{codegray}{rgb}{0.5,0.5,0.5}
\definecolor{codepurple}{rgb}{0.58,0,0.82}
\definecolor{codeblue}{rgb}{0, 0, 204}
\definecolor{backcolour}{rgb}{248,255,235}

\lstdefinestyle{mystyle}{
    backgroundcolor=\color{backcolour},   
    commentstyle=\color{codegreen},
    keywordstyle=\color{codeblue},
    numberstyle=\tiny\color{codegray},
    stringstyle=\color{codepurple},
    basicstyle=\ttfamily\footnotesize,
    breakatwhitespace=false,         
    breaklines=true,                 
    captionpos=b,                    
    keepspaces=true,                 
    numbers=left,                    
    numbersep=5pt,                  
    showspaces=false,                
    showstringspaces=false,
    showtabs=false,                  
    tabsize=2
}

\lstset{style=mystyle}
\usepackage[a4paper, margin = 3cm, bottom = 2.5cm]{geometry}

\newtheorem{theorem}{THEOREM}
\newtheorem{lemma}[theorem]{LEMMA}
\newtheorem{corollary}[theorem]{COROLLARY}
\newtheorem{proposition}[theorem]{PROPOSITION}
\newtheorem{remark}[theorem]{REMARK}
\newtheorem{definition}[theorem]{DEFINITION}
\newtheorem{fact}[theorem]{FACT}

\newcommand{\nats}{\mbox{\( \mathbb N \)}}
\newcommand{\rat}{\mbox{\(\mathbb Q\)}}
\newcommand{\rats}{\mbox{\(\mathbb Q\)}}
\newcommand{\reals}{\mbox{\(\mathbb R\)}}
\newcommand{\ints}{\mbox{\(\mathbb Z\)}}

%%%%%%%%%%%%%%%%%%%%%%%%%%

% ========================================
% Title Page
% ========================================
\title{{\vspace{-14em} \includegraphics[scale=0.4]{Logos/ucl_logo.png}}\\
{{\vspace{2em} \Huge Beyond the Circle: Deforming Contours in Inverse Z-Transform}}\\
{\large Final Year Project Report}\\
}
\date{Submission date: \today}
\author{Roman Ryan Karim\thanks{
{\bf Disclaimer:}
This report is submitted as part requirement for the MEng degree in Mathematical Computation at UCL. It is substantially the result of my own work except where explicitly indicated in the text. The report may be freely copied and distributed provided the source is explicitly acknowledged.}
\\ \\ Dr Carolyn Phelan
\\ \\ \\ \\ Department of Computer Science
\\ University College London
\\ \\
}


% ========================================
% Report
% ========================================
\begin{document}
 
\onehalfspacing
\maketitle
\begin{abstract}

\end{abstract}

% ========================================
% Contents
% ========================================
\tableofcontents
\setcounter{page}{1}

% ========================================
% Introduction
% ========================================
\chapter{Introduction}
\section{Motivation}

\section{Aims and Objectives}

\section{Overview}

% ========================================
% Background
% ========================================
\chapter{Background}
"By definition, a complex number $z$ is an ordered pair ($x, y$) of real numbers $x$ and $y$, written $z = (x, y)$" \cite{kreyszig2010advanced}. In practice, complex numbers are written in the form $z = x + iy$, where $x$ and $y$ are real numbers and $i$ is the imaginary unit. The set of complex numbers is denoted by $\mathbb{C}$.


\section{The \texorpdfstring{$\mathcal{Z}$}{Lg}-Transform}
\todo[inline]{Come up with a more catchy Z-Transform title}

The $z$-transform is a transformation of a real or complex continuous time function $x(t)$, often used for discrete time signals and is commonly described as the discrete time Fourier transform (DTFT). Taking the Fourier transform of a sampled function results in
\begin{flalign}
&& \mathcal{F}\bigg[x(t) \sum^{\infty}_{n = -\infty} \delta(t - n \Delta t)\bigg] &= \int^{\infty}_{-\infty} x(t) \sum_{n = - \infty}^{\infty} \delta (t - n\Delta t)e^{-i \omega t} dt && \\
&& &= \sum_{n = - \infty}^{\infty} \int^{\infty}_{-\infty} x(t) \delta (t - n\Delta t)e^{-i \omega t} dt && \\
&& &= \sum^{\infty}_{n=-\infty} x(n \Delta t)e^{-i \omega nt} &&
\end{flalign}
where we make use of the sifting property of the delta function. If we normalise the sampling interval to 1, we get

\begin{equation}\label{dtft}
	\sum^{\infty}_{n = - \infty} x(n)e^{-i n \omega}
\end{equation}

The sequence $x(n)$ is sampled at discrete time intervals $t_n = n \triangle t$, where the sampling interval $\triangle t$ is the time between consecutive samples and the time index $n$ numbers the samples. The DTFT is a periodic function of $\omega$ with period $2\pi$ and the existence of (\ref{dtft}) relies on the condition of absolute summability of the sequence $x(n)$; all the terms must converge to a finite value. This requirement is mathematically expressed as

\begin{equation}
	\sum^{\infty}_{n = -\infty} |x(n)| < \infty
\end{equation}

We can extend our analysis to the $Z$-transform, which generalizes the discrete time Fourier transform to the complex plane, not just the unit circle where $r = 1$. The $Z$-transform of a sequence $x(n)$ is formally given by

\begin{equation}\label{bilateral_z-transform}
	X(z) = \mathcal{Z}_{n \rightarrow z}[x(n)] = \sum^{\infty}_{n = -\infty} x(n)z^{-n}
\end{equation}

For a complete description of $z$ in the complex plane, we tend to its polar form $z = re^{i\theta}$, where $r$ represents the magnitude of $z$ and $\theta$ (often written as $\omega$ in the context of the unit circle for the DTFT) represents the angle of $z$ with respect to the positive real axis.

In the analysis of causal systems - systems for which a time origin is defined and is illogical to consider signal values for negative time - the unilateral $Z$-transform is used. Unlike the bilateral $Z$-transform in (\ref{bilateral_z-transform}), we sum from zero to positive infinity yielding

\begin{equation}\label{unilateral_z-transform}
	X(z) = \mathcal{Z}_{n \rightarrow z}[x(n)] = \sum^{\infty}_{n = 0} x(n)z^{-n}
\end{equation}

The region within the complex $z$-plane where (\ref{unilateral_z-transform}) converges is known as the Region of Convergence (ROC). The ROC is defined for the set of values of $z$ for which the $Z$-transform is absolutely summable

\begin{equation}\label{roc}
	\textbf{ROC} = \Biggl\{ z : \sum^{\infty}_{n = 0} |x(n)z^{-n}| < \infty \Biggr\}
\end{equation}

The ROC for causal sequences is typically the exterior of the outermost pole in the $Z$-plane, denoted as $|z| > a$. If we say that $z_1$ converges, then $z_1$ exists within the ROC. Thus, all $z$ such that $|z| \geq |z_1|$ also converge. This region excludes the poles themselves, as the transform does not converge at those points. For the system to be \textit{stable}, it's essential that the ROC includes the unit circle, $|z| = 1$. Thus, all poles must lie within the unit circle \cite{LovelessGuido2021}.

\subsection{Probability Generating Functions}\label{pgfs}

\todo[inline]{May be better to move within Abate and Whitt's method}

\section{The Inverse \texorpdfstring{$\mathcal{Z}$}{Lg}-Transform}

The inverse $Z$-transform aims to find the $n$-th value of the sequence $x(n)$ given the $Z$-transform $X(z)$. This is commonly defined as a Cauchy integral around a contour $C$ in the complex plane. The contour $C$ is a counter-clockwise closed path that encloses the region of convergence (ROC). The inverse $Z$-transform is formally given by

\begin{equation}\label{inverse_z-transform}
	x(n) = \mathcal{Z}^{-1}_{z \rightarrow n}[X(z)] = \frac{1}{2\pi i} \oint_C X(z)z^{n-1}dz
\end{equation}

In real-world applications, we often require numerical approximation due to computational challenges posed by the Cauchy integral formula. Such approximations enable the effective analysis and processing of complex signals within various technological and financial systems.
\newline
\todo[inline]{think about how to make a smooth transition from IZT to NIZT (and how we only discuss contour integration methods given the nature of this project)}

\subsection{Abate and Whitt 1992}
The numerical approximation formula offered by Abate and Whitt is based upon a Fourier series \cite{AbateWhitt1992a, AbateWhitt1992b} catering to the inversion of probability generating functions as elucidated in Section \ref{pgfs}. The format is conducive to queueing theory and other probabilistic models where the $Z$-transform is defined as $q = 1 / z$. The authors approximate (\ref{inverse_z-transform}) using a trapezoidal rule for numerical integration over a complex contour

\begin{equation}\label{aw_inversion}
	x(n) \approx \frac{1}{2nr^n} \biggr( X(r) + 2\sum^{n}_{k = 1} (-1)^k \text{Re}(X(re^{\frac{ik\pi}{n}})) + (-1)^nX(-r) \biggl)
\end{equation}

The parameter $r$ is used to control the error; setting $r = 10^{-\lambda / 2n}$ yields an accuracy bound of $10^{-\lambda}$. The authors leverage the inherent symmetry within the complex plane to enhance computational efficiency. Specifically exploiting the complex conjugate symmetry of $X(z)$, each term $X(re^{\frac{ik\pi}{n}})$ in the upper half has a mirror image in the lower half. By \textit{folding} the problem in this manner, the computational load is effectively halved.

It is also important to note that the following changes adhere to the principles set forth by the Nyquist-Shannon sampling theorem.

\subsection{Cavers 1978}

\subsection{Series acceleration techniques}

\section{Discrete Pricing Options}

\subsection{Lookback and Barrier Options}

\section{Optimization Techniques}

\subsection{Stochastic Gradient Descent}

\subsection{Bayesian Optimization}

% ========================================
% Experiment
% ========================================
\chapter{Experiment}

\todo[inline]{Finding different parameters to use for the experiment making use of Machine Learning techniques.}

% ========================================
% Results
% ========================================
\chapter{Results}

% ========================================
% Conclusion
% ========================================
\chapter{Conclusion}
\section{Summary}

\section{Future Work}

\section{Acknowledgements}


% ========================================
% References 
% ========================================
\addcontentsline{toc}{chapter}{References}
\bibliography{references}
\bibliographystyle{apalike}

% ========================================
% Appendix
% ========================================
\begin{appendices}

\chapter{Initial Project Plan}

\includepdf[pages=-]{initial_project_plan.pdf}
    
\end{appendices}

\end{document}